%% The first command in your LaTeX source must be the \documentclass command.
%%
%% Options:
%% twocolumn : Two column layout.
%% hf: enable header and footer.
\documentclass[
twocolumn,
% hf,
]{ceurart}

%%
%% One can fix some overfulls
\sloppy

%%
%% Minted listings support 
%% Need pygment <http://pygments.org/> <http://pypi.python.org/pypi/Pygments>
\usepackage{listings}
\usepackage{bm}
\usepackage{url}

%% auto break lines
\lstset{breaklines=true}

%%
%% end of the preamble, start of the body of the document source.
\begin{document}

%%
%% Rights management information.
%% CC-BY is default license.
\copyrightyear{2022}
\copyrightclause{Copyright for this paper by its authors.
  Use permitted under Creative Commons License Attribution 4.0
  International (CC BY 4.0).}

%%
%% This command is for the conference information
\conference{COMHUM 2022: Workshop on Computational Methods in the Humanities,
  June 9--10, 2022, Lausanne, Switzerland}

%%
%% The "title" command^
\title{Refining character relationships using embeddings of textual units}

%%
%% The "author" command and its associated commands are used to define
%% the authors and their affiliations.
\author[1]{Guillaume Guex}[%s
orcid=0000-0003-1001-9525,
email=guillaume.guex@unil.ch]
\address[1]{Departement of Language and Information Sciences, University of Lausanne, Switzerland}

%%
%% The abstract is a short summary of the work to be presented in the
%% article.
\begin{abstract}
  A clear and well-documented \LaTeX{} document is presented as an
  article formatted for publication by CEUR-WS in a conference
  proceedings. Based on the ``ceurart'' document class, this article
  presents and explains many of the common variations, as well as many
  of the formatting elements an author may use in the preparation of
  the documentation of their work.
\end{abstract}

%%
%% Keywords. The author(s) should pick words that accurately describe
%% the work being presented. Separate the keywords with commas.
\begin{keywords}
  LaTeX class \sep
  paper template \sep
  paper formatting \sep
  CEUR-WS
\end{keywords}

%%
%% This command processes the author and affiliation and title
%% information and builds the first part of the formatted document.
\maketitle

\section{Introduction}

Distant reading tools allow researchers, from various fields, to quickly gain knowledge on textual corpora without actually reading them. Purposes of these methods are various, but can be mainly categorized into two groups: in the first case, these methods are used in order to tag, classify, or summary large quantities of texts, in order to quickly structure information or to deliver a speech over the whole studied corpus. Methods in this case rely heavily on Big Data and make an extensive use of Machine Learning algorithms. In the second case, researchers use these methods to underline hidden structures in a particular text, helping them to refine their understanding of it and reinforce stated hypotheses. Methods in this setting can also rely on Machine Learning, but must typically be build with more caution and attention to details: corpus are smaller, analyses are closer to the work, and methods must be more transparent in order to appropriately interpret results.

Automatic extraction and analysis of \emph{character networks} from literacy works typically belong in the latter group. These methods aim at representing various interactions occurring between fictional characters found in a textual narrative with a graph, thus showing explicitly the hidden structure of character relationships constructed by the author. This structure might allow to find hidden patterns within a book, which can highlight a particular genre or style.

\section{Methodology}


When building character networks from a textual narrative, the most widespread method consists in dividing the studied work into $n$ textual units $u_1, \ldots, u_n$, which can be, e.g., sentences, paragraphs, or chapters, and then counting characters co-occurrences in these units. Usually, the text constituting these units is discarded and the resulting network displays edges which roughly represent an aggregated number of interactions between characters. However, by doing so, the aggregation occurs on various type of interactions and will give little information about the type of relationship which exist between characters. In this paper, we propose a data organization leading to various methods which takes into consideration the text contained in the unit, helping to refine understanding of characters and their relationships.

\subsection{Data organization}

In this article, the data representation for a textual narrative, divided in $n$ textual units $u_1, \ldots, u_n$, is made through two tables. The first one is well known in the field of textual analysis, and consists in the $(n \times v)$ contingency table $\mathbf{N}$, as represented by Table \ref{cont_table}, where $v$ is the vocabulary size. In this table, each row represents an unit, each column a word, and cells $n_{ij}$ counts the number of times the word $i$ appears in the unit. Using this table typically denotes a \emph{Bag-of-Words} approach in our analyses.

\begin{table}[h]
	\scriptsize
	\begin{tabular}{|c||c|c|c|c|c|c|}
		\hline
		& aller & allumer & apercevoir & bas & bon & ... \\
		\hline
		\hline 
		$u_{100}$ & 23 & 2 & 6 & 11 & 6 & ... \\
		\hline
		$u_{7795}$ & 12 & 1 & 0 & 3 & 9 & ... \\
		\hline
		$u_{7796}$ & 10 & 0 & 5 & 1 & 5 & ... \\
		\hline
		$u_{7797}$ & 0 & 0 & 1 & 0 & 0 & ... \\
		\hline
	\end{tabular}
	\label{cont_table}
	\caption{A snippet of the contingency table $\mathbf{N}$ extracted from \emph{les Misérables}. Rows are chapters, columns are words in the vocabulary, and cell $n_{ij}$ counts the number of time word $j$ appear in chapter $i$.}
\end{table}

The second table, denoted by $\mathbf{O}$, has a size of $(n \times p)$ where $p$ is the number of \emph{character-based objects} found in the narrative and cells $o_{ij}$ counts occurrences of object $j$ in unit $i$. A character-based object can be loosely defined in order to be flexible for various types of narrative or analyses, but can be roughly seen as a countable recurring narrative entity, containing one or more characters. For example, it can be a character occurrence, a character co-occurrence, an oriented character interaction (e.g. a dialog), or even a particular recurring event containing multiple characters (e.g. a meeting). In this article, we mostly consider character occurrences and pair of characters co-occurrences, as shown in Table \ref{occ_table}. 

\begin{table}[h]
	\scriptsize
	\begin{tabular}{|c||c|c|c|c|}
		\hline
		& Cosette & Thénardier & Valjean & ... \\
		\hline
		\hline 
		$u_{101}$ & 66 & 78 & 0 & ... \\
		\hline
		$u_{102}$ & 21 & 26 & 0 & ... \\
		\hline
		$u_{103}$ & 12 & 25 & 0 & ... \\
		\hline
		$u_{104}$ & 12 & 0 & 5 & ... \\
		\hline
	\end{tabular}
	\begin{tabular}{|c||c|c|c|c|}
		\hline
		& ... & Cosette-Thénardier & Cosette-Valjean & ... \\
		\hline
		\hline 
		$u_{101}$  	& ... & 66 & 0 &  ... \\
		\hline
		$u_{102}$ 	& ... & 21 & 0 & ... \\
		\hline
		$u_{103}$  	& ... & 12 & 0 & ... \\
		\hline
		$u_{104}$ 	& ... & 0 & 5 & ... \\
		\hline
	\end{tabular}
	\label{occ_table}
	\caption{A snippet of the character-based objects table $\mathbf{O}$ extracted from \emph{les Misérables}. Rows are chapters, columns are character occurrences (top) and character co-occurrences (bottom), and cell $o_{ij}$ counts the number of times object $j$ appear in chapter $i$. Character co-occurrences are computed here as $o_{ij} = \min(o_{ik}, o_{il})$, where $\{k, l\}$ are characters constituting $j$.}
\end{table}

This data organization already gives an orientation to the subsequent analyses and should be kept in mind of the practitioner. Textual unit are now considered as \emph{individuals} (in the statistical terminology), defined by their \emph{variables} contained in the different columns of both tables. Moreover, subsequent analyses are oriented in searching how the object table $\mathbf{O}$ has an influence over the contingency table $\mathbf{N}$, i.e. searching which words are over-represented or under-represented knowing the character-based objects in the unit. While authors use characters in order to build the narrative, we, to a certain extent, work backward: we are searching how character appearances and interactions in the textual unit act on her/his choice of words. If the extraction method permits it, a practitioner should include all character-based objects which she/he desire to study. Here, for example, the choice to include character co-occurrences along with occurrences is motivated by the fact that we are interested in studying character relationships. A character co-occurrence can roughly be seen as an interaction between two characters, and an interaction between two characters (or more, but higher order interactions are outside the scope of this article) should be considered as an object of its own: this interaction is not necessarily the sum of its parts and can give a particular flavor to the unit.

This data organization also highlight the importance of choosing a proper size for the units. These units should be large enough to contain enough words in order the properly capture the textual specificity of each unit, but not too large, as each unit should ideally captures particularities about one of the character-based objects. Unfortunately, it is impossible to define an ideal size for all types of analysis and this size should be balanced regarding the level of analysis, text size, selected character-based objects, and previous knowledge of the studied work.

The use of a contingency table to represent the textual resource present in the units denotes a \emph{Bag-of-Words} approach. Using this approach loses the information relative to the order of words in the units, but permits to transform a chain of characters, improper to statistical analyses, into a contingency table, a well studied mathematical object which permits the use of various kind of statistical methods. The next section shows two methods of analysis based on this table.

\subsection{Embedding of textual units}

Various methods can be performed on the contingency table $\mathbf{N}$ in order to extract information from it, such as Sentiment Analysis (REF), Textometry (REF), or even Deep Learning methods. Here, we make to choice to extract a lower dimensional, numeric representation of each units, in other words, an \emph{embedding of textual units}. \\
In section (REF), these vectors of textual units are used as anchor points in order to also embed character-based objects in the same space. Therefore, it is crucial that an interpretation about the directions, or the regions, of this embedding is possible in order to properly extract information about the localization of character-based objects vectors (the relative position of these vectors is insufficient). For that reason, we focus on embeddings of textual units which also contain \emph{vectors of words}: by examining the positions of character-based objects relatively to word vectors, these objects can be characterized.

We propose here two embeddings verifying this condition: Section (REF) describes \emph{Correspondence Analysis (CA)} and section (REF) focuses on \emph{Pre-trained word vectors (WV)}. 

\subsubsection{Correspondence Analysis (CA)}

Using Correspondence Analysis in order to analyze textual resources has a long tradition in literature (REF). It has the advantage to naturally provide an embedding space, the factorial map, where units are placed alongside word vectors, allowing to analyze or represent the different units in terms of word frequency profiles. Moreover, units and words placement in the embedding space have a direct interpretation in terms of chi2 distance between profiles, which is desirable when interpreting results.

By performing a Correspondence Analysis on table $\mathbf{N}$, we get $n$ vectors $\mathbf{x}_1, \ldots, \mathbf{x}_n$ corresponding to units (rows) and $v$ vectors $\mathbf{w}_1, \ldots, \mathbf{w}_v$ corresponding to words (columns). Each of these vectors has a size of $\min(n, v) - 1$, which will generally be $n - 1$. For detailed computations of quantities given by the CA, see Appendix \ref{ca_details}. The position of a unit vector $\mathbf{x}_i$ can be interpreted by mutliple approaches, but we mainly use two here.

The first way to interpret it is to look at the unit-vector position on a particular axis $\alpha$, i.e. looking at the component $x_{i\alpha}$. This component gives the position of the unit on a particular found "contrast" between units, depending on the chosen $\alpha$, and this contrast will be strong if $p_\alpha$, i.e. \emph{the proportion of inertia expressed in $\alpha$}, is high. It can further be interpreted by observing the \emph{contribution of each word $j$} composing the axis $\alpha$, i.e. $c^w_{j\alpha}$ and looking at the sign of $w_{j\alpha}$. Hopefully, by extracting and interpreting the most positive and negative contributing words, the chosen axis can express a particular duality which operates inside the studied text (e.g. love-hatred, action-description, lightness-darkness). The second method is to directly look at the \emph{similarities} $s_{ij}$ between an unit $i$ and a word $j$, as defined by the scalar product between their vectors $s_{ij} := \mathbf{x}^\top_i \mathbf{w}_j$. A positive (resp. negative) similarity denotes a over-representation (resp. under-representation) of the word $j$ in $i$, which permit to find lists of words characterizing the different units. Observe that in this article, both methods are rather applied to character-based object vectors defined in section (REF), as they lay in the same space as unit vectors. 

Using CA to embedded units however comes with some limitations. First, note that vectors $\mathbf{x}_1, \ldots, \mathbf{x}_n$ obtained from CA reflect textual unit profile (in terms of words) regarding the mean profile. This analysis is therefore \emph{contrastive}: it highlights unit variations among the studied text. This could becomes problematic in order to study a dataset containing multiple texts: units belonging to the same text will have a greater chance to be close from each other, and the displayed variance might be largely composed by differences of style between text. Another limitation with this approach is that the words helping the interpretation of units (and character-based objects) are contained in the studied text. Approaches requiring to study the position of units and objects relatively to a predefined list of words (e.g., friends, enemies, family) might therefore be impossible.

\subsubsection{Pre-trained word vector (WV)}

Pre-trained word vectors, based on Word2Vec (REF), Glove (REF), or fastText (REF), have recently received great attention from various fields (REFS). They are generally obtained through a training on a very large corpora, such as Wikipedia (REF) or Common Crawl (REF), and the resulting embedding contains a large quantity of word vectors. As shown by multiple studies (REF), these vectors are placed in order to reflect semantic and syntactic similarities between words, and are used in mutliple applications (REF). 

There exists various ways to use these pre-trained word vectors in order to find vectors for a \emph{group of words}, such as sentences (REF), paragraphs (REF), or documents (REF), which are often used to apply a classification or clustering algorithm on the newly embedded objects (REF). Here, we use a recent methodology proposed in (REF), in order to obtain textual unit vectors laying in the same space as words. 

More precisely, we use a pre-trained word vectors $w_1, \ldots, w_m$ trained on Common Crawl and Wikipedia using fastText.\footnote{\url{ https://fasttext.cc/docs/en/crawl-vectors.html}, accessed September 2022.} For French, the number of word-vectors $m$ is 

\subsection{Character-based object embeddings}


\section{Results}

\section{Conclusion}

\appendix

\section{Appendix}

\subsection{Correspondence Analysis}
\label{ca_details}

Starting from the $(n \times v)$ contingency table $\mathbf{N} = (n_{ij})$, we define the vector of unit weights as $\mathbf{f} = (f_i) := (n_{i\bullet}/n_{\bullet \bullet})$ and the vector of word weights as $\mathbf{g} = (g_j) := (n_{\bullet j}/n_{\bullet \bullet})$, where $\bullet$ denotes the summation on the replaced index. It is then possible to compute the \emph{weighted scalar product matrix between units} $\mathbf{K} = (k_{ij})$ with

\begin{equation}
k_{ij} := \sqrt{f_i f_j} \sum_{k=1}^{v} g_k(q_{ik} - 1)(q_{jk} - 1), 
\end{equation}

where $q_{ik} = \frac{n_{ik} n_{\bullet \bullet}}{n_{j \bullet} n_{\bullet k}}$ is the \emph{quotient of independence} of the cell $i, k$. The vector of textual unit $i$, $\mathbf{x}_i = (x_{i\alpha})$, is obtained by the eigendecomposition of the matrix $\mathbf{K} = \mathbf{U}\bm{\Lambda}\mathbf{U}^\top$ and with

\begin{equation}
x_{i\alpha} := \frac{\sqrt{\lambda_\alpha}}{\sqrt{f_i}} u_{i \alpha},
\end{equation}

where $\lambda_\alpha$ are the eigenvalues contained in the diagonal matrix $\bm{\Lambda}$ and $u_{i \alpha}$ the eigenvectors components found in $\mathbf{U}$. We find the vector of word $j$, $\mathbf{w}_j = (w_{j\alpha})$, with 

\begin{equation}
w_{j\alpha} := \frac{1}{\sqrt{\lambda_\alpha}} \sum_{i=1}^v f_i q_{ij} x_{i \alpha}.
\end{equation}

Note that various other quantities of interest can also be computed in CA, such as 
\begin{align*}
p_\alpha &:= \frac{\lambda_\alpha}{\lambda_\bullet} :\text{\small the proportion of inertia expressed in $\alpha$,} \\
c^u_{i \alpha} &:= \frac{f_i x_{i\alpha}^2}{\lambda_\alpha} : \text{\small the contribution of unit $i$ to axis $\alpha$,} \\
c^w_{j \alpha} &:= \frac{g_j w_{j\alpha}^2}{\lambda_\alpha} : \text{\small the contribution of word $j$ to axis $\alpha$,} \\
h^u_{i \alpha} &:= \frac{x_{i\alpha}^2}{\sum_{\alpha} x_{i\alpha}^2} : \text{\small the contribution of  axis $\alpha$ to unit $i$,} \\
h^w_{j \alpha} &:= \frac{w_{j\alpha}^2}{\sum_{\alpha} w_{j\alpha}^2} : \text{\small the contribution of  axis $\alpha$ to word $j$,}
\end{align*}
For a detailed interpretation of these different quantities, see (REF). 


%%
%% Define the bibliography file to be used
\bibliography{charnet}

\end{document}

%%
%% End of file
